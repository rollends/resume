\documentclass[oneside, 10pt]{memoir}
\usepackage[
    usefilenames,% Important for XeLaTeX
    RMstyle={Text,Semibold},
    SSstyle={Text,Semibold},
    TTstyle={Text,Semibold},
    DefaultFeatures={Ligatures=TeX}
]{plex-otf}
\usepackage{hyperref}
\usepackage{longtable}
\usepackage{url}


% ---------------------------------------------------------------------------------------------------------------------
% General Styling
%

\pagestyle{plain}
\makeevenfoot{plain}{}{}{}
\makeoddfoot{plain}{}{}{}

\setlength{\parindent}{0em}
% \setlength{\parskip}{0em}
% \setlength{\partopsep}{0.25em}

% \setheadfoot{1.5em}{1em}
\setlrmarginsandblock{4cm}{4cm}{*}
\setulmarginsandblock{1.5cm}{1.5cm}{*}
\checkandfixthelayout

\setbeforesecskip{2em}
\setaftersecskip{1em}
\setsecheadstyle{\Large\centering\bfseries}
\setsubsecheadstyle{\bfseries\slshape}
\def\arraystretch{1.45}%

% ---------------------------------------------------------------------------------------------------------------------
% Document
% 

% EDUCATION ENTRY COMMAND
%
\begin{document}
% General Entry Command
% #1 : Flush Right (Heading)
% #2 : Flush Left (Heading)
% #3 : Subtitle
% #4 : Body
\newcommand{\entryGeneral}[4]{
    \textbf{#2} \sourceatright{#1}
    \emph{#3}
    \begin{adjustwidth}{1em}{0em}
        #4
    \end{adjustwidth}
    \hfill
}

% Simple Entry Command (no subtitle)
% #1 : Flush Right (Heading)
% #2 : Flush Left (Heading)
% #3 : Body
\newcommand{\entrySimple}[3]{
    \textbf{#2} \sourceatright{#1}
    \begin{adjustwidth}{1em}{0em}
        #3
    \end{adjustwidth}
    \hfill
}

% ---------------------------------------------------------------------------------------------------------------------
% Begin Resume
%
\LARGE{{Rollen S. D'Souza}}\\
\small{\texttt{rollen.dsouza@uwaterloo.ca}~\textbullet~\href{https://github.com/rollends}{\texttt{github.com/rollends}}~\textbullet~\href{https://rollends.ca}{\texttt{rollends.ca}} }\\
\rule{\linewidth}{0.4pt}

\section*{Education}

\begin{tabular}{ll}
    {2022}
        & \emph{Candidate for} Doctor of Philosophy\\
        & \emph{Exterior Differential Characterization for Transverse Feedback Linearization}\\
        & Supervisor: Prof. Christopher Nielsen\\
        & University of Waterloo.\\
        & \\
    {2017}
        & Bachelor of Software Engineering\\
        & Honours Software Engineering\\
        & Joint Honours Applied Mathematics\\
        & University of Waterloo.
\end{tabular}

\section*{Teaching Experience}
Experienced educator having experience ranging from grading assignments and delivering tutorials to designing assessments and lab content.
Knowledge spans topics in software engineering to control theory.

\begin{longtable}{l|l|l}
    \textbf{Term} & \textbf{Course} & \textbf{Class Size}\\ \hline\hline
    & & \\\hline
    \multicolumn{3}{c}{\textbf{Lab Instructor}}\\\hline
    & & \\
    {2020 FALL  } & {SE 380 (Introduction to Control Theory)} & 2\\
    {2020 SPRING} & {SE 380 (Introduction to Control Theory)} & 2\\
    {2018 WINTER} & {SE 350 (Operating Systems)} & 2\\
    & & \\\hline
    \multicolumn{3}{c}{\textbf{Teaching Assistant}}\\\hline
    & & \\
    {2021 WINTER} & {MATH 213 (Signals and Systems)} & 115\\
    {2020 WINTER} & {ECE 380 (Introduction to Control Theory)} & 142\\
    {2019 FALL  } & {SE 380 (Introduction to Control Theory)} & 91\\
    {2019 SPRING} & {ECE 351 (Compilers)} & 162\\
    {2019 WINTER} & {ECE 459 (Programming for Performance)} & \\
    {2018 FALL  } & {SE 380 (Introduction to Control Theory)} & 109\\
    {2018 SPRING} & {MATH 213 (Signals and Systems)} & 118\\
    {2017 FALL  } & {SE 101 (Introduction to Software Engineering)} & 139\\
    {2017 WINTER} & {SE 465 (Software Testing and Quality Assurance)} & \\
    {2016 FALL  } & {SE 101 (Introduction to Software Engineering)} & 131
\end{longtable}

\section*{Academic Service}

\section*{Academic Publications}

\subsection*{Journal Publications}
R. S. D'Souza and C. Nielsen. \emph{An Exterior Differential Characterization of Single-Input Local Transverse Feedback Linearization}, Automatica. [To Appear]\hspace*{\fill}\\[1em]
R. S. D'Souza, R. Louwers, C. Nielsen. \emph{Piecewise-Linear Path Following for a Unicycle using Transverse Feedback Linearization}, IEEE Transactions on Control Systems Technology,\hspace*{\fill}\linebreak DOI: \href{https://dx.doi.org/10.1109/TCST.2021.3049715}{10.1109/TCST.2021.3049715}. [To Appear]\hspace*{\fill}

\subsection*{Conference Publications}
R. S. D'Souza and C. Nielsen, \emph{Piecewise-Linear Path Following for a Unicycle using Transverse Feedback Linearization}, American Control Conference, 2020.\hspace*{\fill}\\[1em]
R. S. D'Souza and C. Nielsen, \emph{Dual Conditions for Local Transverse Feedback Linearization}, IEEE Conference on Decision and Control, 2018.\hspace*{\fill}\\[1em]
V. Joukov, R. D'Souza and D. Kuli\'{c}, \emph{Human pose estimation from imperfect sensor data via the Extended Kalman 
Filter}, International Symposium on Experimental Robotics, 2016.\hspace*{\fill}

\begin{comment}
\section*{Detailed Teaching Experience}
\begin{adjustwidth}{1em}{0em}%
    \entrySimple{2021/01\textemdash Present}{Teaching Assistant}{
        A teaching assistant for Software and Electrical \& Computer Engineering courses:
        \begin{itemize}
            \item{MATH 213 : Signals and Systems}
        \end{itemize}
        Duties included drafting and delivering tutorials online through the \texttt{Microsoft Teams} environment as well as drafting useful supplementary course material for additional practice offline.
    }

    \entrySimple{2020/05\textemdash 2020/12}{Lab Instructor}{
        Developed and administered new, remote learning based, lab content for a Control Theory course using \texttt{MATLAB}.
        Recorded and delivered online lab tutorials through \texttt{Youtube} and engaged directly with students through the use of \texttt{Discord}.
        Successfully delivered lab content for two successive terms.
        Lab content continues to be used by other lab instructors while online, remote learning persists.
    }

    \entrySimple{2019/05\textemdash 2020/04}{Teaching Assistant}{
        A teaching assistant for Software and Electrical \& Computer Engineering courses:
        \begin{itemize}
            \item{ECE 351 : Compilers}
            \item{SE/ECE 380 : Introduction to Feedback Control}
        \end{itemize}
        Duties included drafting and delivering tutorials and grading assessments.
        Helped improve and maintain a \texttt{MATLAB} project for SE/ECE 380.
    }

    \entrySimple{2019/01\textemdash 2019/04}{Lab Instructor}{
        Coordinated the Operating Systems (SE 350) course Lab component.
        Students, in groups of four, had to build a basic real-time operating system starting with skeleton code.
        Led a team of two teaching assistants in helping and grading students.
        Provided direct hands-on help with understanding systems and debugging complex programming and hardware-related bugs.
    }

    \entrySimple{2017/09\textemdash 2018/12}{Teaching Assistant}{
        A teaching assistant for Software-related courses:
        \begin{itemize}
            \item{SE 101 : Introduction to Software Engineering}
            \item{ECE 459 : Programming for Performance}
            \item{MATH 213 : Signals and Systems for Software Engineers}
            \item{SE 380 : Introduction to Feedback Control}
        \end{itemize}
        Duties included drafting and delivering tutorials, grading assessments and guiding students through independent group projects.
        Provided 1-1 guidance for students on independent group projects for SE 101, helping develop and grow their ability to design and develop software.
    }

    \entrySimple{2016/09\textemdash 2017/04}{Undergraduate Teaching Assistant}{
        A teaching assistant for Software-related courses:
        \begin{itemize}
            \item{SE 101 : Introduction to Software Engineering}
            \item{SE 465 : Software Testing and Quality Assurance}
        \end{itemize}
        Duties included managing a weekly 2-hour lab section, marking weekly quiz submissions as well as a final course project. 
        Engaged with students through Piazza and office hours.
    }
\end{adjustwidth}
\end{comment}

\section*{Honours and Awards}
\begin{center}
\begin{tabular}{lll}
    \textbf{Year} & \textbf{Agency} & \textbf{Title}\\ \hline\hline
    2020    &   Govt. of Ontario & QEII Grad. Scholar. Science \& Tech.\\
    2018
        & \begin{minipage}[t]{18em} University of Waterloo\\ Sir Sanford Fleming Foundation \end{minipage}
        & TA Excellence Award\\
    2018
        & \begin{minipage}[t]{18em} University of Waterloo\\ Dept. of Electrical \& Computer Eng. \end{minipage}
        & TA Award\\
    2016    &   NSERC & Undergrad Student RA (USRA) \\
    2016
        & \begin{minipage}[t]{18em} University of Waterloo\\ Software Eng. Program \end{minipage}
        & General Motors (GM) Innovation Award \\
    2015    &   NSERC & Undergrad Student RA (USRA) \\
    2014
        & \begin{minipage}[t]{18em} International Genetically Engineered\\ Machine (iGEM) Competition \end{minipage}
        & Best Model \\
    2011    &   University of Waterloo & President's Scholarship \\
\end{tabular}
\end{center}

% \section*{Other Activities}
% \begin{adjustwidth}{1em}{0em}%
%     \entryGeneral{2018/09\textemdash 2019/12}{Software Engineering Mentor}
%     {University of Waterloo, Program of Software Engineering}{
%         Part-time mentor for first year Software Engineering students in all courses that are deemed at need for a student.
%     }

%     \entryGeneral{2015/09\textemdash 2016/04}{Software Engineering Curriculuum Committee Student Representative}
%     {University of Waterloo, Program of Software Engineering}{
%         Represented student body on matters regarding curriculuum changes for future classes of Software Engineering.
%     }

%     \entryGeneral{2015/01\textemdash 2015/12}{Math Modeller}
%     {International Genetically Engineered Machine (iGEM) Competition - Team Waterloo} {
%         Developed and analyzed mathematical models of the CRISPR-mRNA interference mechanism alongside
%         interdisciplinary team of mathematics, physics, engineering and biology undergraduate students.
%         Main contribution consisted of performing global sensitivity analysis.
%     }

%     \entryGeneral{2015/01\textemdash 2015/12}{Volunteer First Responder}{Campus Response Team, FEDS} {
%         Provided first aid services at student events and the student activity complex. 
%         Certified Standard First Aid/CPR-HCP.
%     }
% \end{adjustwidth}


% \section*{Projects}
% \begin{adjustwidth}{1em}{0em}%
%     \entryGeneral{2017/01\textemdash 2017/04}{Project: Comparison of Transverse Feedback Linearizing Controllers}{SE499, University of Waterloo}{
%         Undergraduate research project investigating two different implementations of path following controllers. Supervised by Prof. Christopher Nielsen.
%     }

%     \entryGeneral{2016/01\textemdash 2016/04}{Project: Real Time Operating System}{CS452, University of Waterloo}{
%         Implemented (hard) real time operating system in C, from the ground up, for the purpose of autonomously
%         managing trains on a physical track.
%     }
% \end{adjustwidth}


\end{document}