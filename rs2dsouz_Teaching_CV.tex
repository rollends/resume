\documentclass[oneside, 10pt]{memoir}
\usepackage{fontspec}
\usepackage{url}
\usepackage{hyperref}

%\setmainfont[Ligatures=TeX]{Lora}
%\setmonofont[Ligatures=TeX]{UbuntuMono}
% ---------------------------------------------------------------------------------------------------------------------
% General Styling
%

\pagestyle{plain}
\makeevenfoot{plain}{}{}{}
\makeoddfoot{plain}{}{}{}

\setlength{\parindent}{0em}
\setlength{\parskip}{0em}
\setlength{\partopsep}{0.25em}

\setheadfoot{1.5em}{1em}
\setlrmarginsandblock{3cm}{3cm}{*}
\setulmarginsandblock{3cm}{3cm}{*}
\checkandfixthelayout

\setaftersecskip{1em}
\setbeforesecskip{1em}

% ---------------------------------------------------------------------------------------------------------------------
% Document
% 

\begin{document}

% General Entry Command
% #1 : Flush Right (Heading)
% #2 : Flush Left (Heading)
% #3 : Subtitle
% #4 : Body
\newcommand{\entryGeneral}[4]{
    \textbf{#2} \sourceatright{#1}
    \emph{#3}
    \begin{adjustwidth}{1em}{0em}
        #4
    \end{adjustwidth}
    \hfill
}

% Simple Entry Command (no subtitle)
% #1 : Flush Right (Heading)
% #2 : Flush Left (Heading)
% #3 : Body
\newcommand{\entrySimple}[3]{
    \textbf{#2} \sourceatright{#1}
    \begin{adjustwidth}{1em}{0em}
        #3
    \end{adjustwidth}
    \hfill
}

% ---------------------------------------------------------------------------------------------------------------------
% Begin Resume
%

\LARGE{\textsc{Rollen S. D'Souza}}\\
\small{\url{rollen.dsouza@uwaterloo.ca}~\textbullet~\url{https://github.com/rollends}~\textbullet~\url{https://rollends.ca} }\\
\rule{\linewidth}{0.4pt}

\section*{Education}
\begin{adjustwidth}{1em}{0em}%
    \entryGeneral{Expected 2022/05}{Ph.D \quad (Candidate)\quad(CGPA: \(89.00\))}{University of Waterloo}{
        Thesis studying the application of the exterior differential calculus to nonlinear control design.
        Supervised by Prof. Christopher Nielsen.
    }

    \entryGeneral{2017/07}{BSE\quad(CGPA: \(77.35\))}{University of Waterloo}{
        Honours Software Engineering, Joint Honours Applied Mathematics.
    }
\end{adjustwidth}

\section*{Teaching Summary}
\begin{adjustwidth}{1em}{0em}%
    Experienced educator having experience ranging from grading assignments and delivering tutorials to designing assessments and lab content.
    Knowledge spans topics in software engineering to control theory.
    %
    \subsection*{Lab Instructor}
    \begin{itemize}
        \item{SE 380 : Introduction to Feedback Control (2 Terms)}
        \item{SE 350 : Operating Systems (1 Term)}
    \end{itemize}

    \subsection*{Teaching Assistant}
    \begin{itemize}
        \item{MATH 213 : Signals and Systems (2 Terms)}
        \item{SE/ECE 380 : Introduction to Feedback Control (3 Terms)}
        \item{ECE 351 : Compilers (1 Term)}
        \item{ECE 459 : Programming for Performance (1 Term)}
        \item{SE 465 : Software Testing and Quality Assurance (1 Term)}
        \item{SE 101 : Introduction to Software Engineering (2 Terms)}
    \end{itemize}
\end{adjustwidth}

\section*{Detailed Teaching Experience}
\begin{adjustwidth}{1em}{0em}%
    \entrySimple{2021/01\textemdash Present}{Teaching Assistant}{
        A teaching assistant for Software and Electrical \& Computer Engineering courses:
        \begin{itemize}
            \item{MATH 213 : Signals and Systems}
        \end{itemize}
        Duties included drafting and delivering tutorials online through the \texttt{Microsoft Teams} environment as well as drafting useful supplementary course material for additional practice offline.
    }

    \entrySimple{2020/05\textemdash 2020/12}{Lab Instructor}{
        Developed and administered new, remote learning based, lab content for a Control Theory course using \texttt{MATLAB}.
        Recorded and delivered online lab tutorials through \texttt{Youtube} and engaged directly with students through the use of \texttt{Discord}.
        Successfully delivered lab content for two successive terms.
        Lab content continues to be used by other lab instructors while online, remote learning persists.
    }

    \entrySimple{2019/05\textemdash 2020/04}{Teaching Assistant}{
        A teaching assistant for Software and Electrical \& Computer Engineering courses:
        \begin{itemize}
            \item{ECE 351 : Compilers}
            \item{SE/ECE 380 : Introduction to Feedback Control}
        \end{itemize}
        Duties included drafting and delivering tutorials and grading assessments.
        Helped improve and maintain a \texttt{MATLAB} project for SE/ECE 380.
    }

    \entrySimple{2019/01\textemdash 2019/04}{Lab Instructor}{
        Coordinated the Operating Systems (SE 350) course Lab component.
        Students, in groups of four, had to build a basic real-time operating system starting with skeleton code.
        Led a team of two teaching assistants in helping and grading students.
        Provided direct hands-on help with understanding systems and debugging complex programming and hardware-related bugs.
    }

    \entrySimple{2017/09\textemdash 2018/12}{Teaching Assistant}{
        A teaching assistant for Software-related courses:
        \begin{itemize}
            \item{SE 101 : Introduction to Software Engineering}
            \item{ECE 459 : Programming for Performance}
            \item{MATH 213 : Signals and Systems for Software Engineers}
            \item{SE 380 : Introduction to Feedback Control}
        \end{itemize}
        Duties included drafting and delivering tutorials, grading assessments and guiding students through independent group projects.
        Provided 1-1 guidance for students on independent group projects for SE 101, helping develop and grow their ability to design and develop software.
    }

    \entrySimple{2016/09\textemdash 2017/04}{Undergraduate Teaching Assistant}{
        A teaching assistant for Software-related courses:
        \begin{itemize}
            \item{SE 101 : Introduction to Software Engineering}
            \item{SE 465 : Software Testing and Quality Assurance}
        \end{itemize}
        Duties included managing a weekly 2-hour lab section, marking weekly quiz submissions as well as a final course project. 
        Engaged with students through Piazza and office hours.
    }
\end{adjustwidth}

\section*{Publications}

\subsection*{Journal Publications}
\begin{adjustwidth}{1em}{0em}%
R.S. D'Souza and C. Nielsen. \emph{An Exterior Differential Characterization of Single-Input Local Transverse Feedback Linearization}, Automatica. [To Appear]\hfill\\
\hfill \\
R.S. D'Souza, R. Louwers, C. Nielsen. \emph{Piecewise-Linear Path Following for a Unicycle using Transverse Feedback Linearization}, IEEE Transactions on Control Systems Technology, DOI: \href{https://dx.doi.org/10.1109/TCST.2021.3049715}{10.1109/TCST.2021.3049715}. [To Appear]\hfill
\end{adjustwidth}

\subsection*{Conference Publications}
\begin{adjustwidth}{1em}{0em}%
R. S. D'Souza and C. Nielsen, \emph{Piecewise-Linear Path Following for a Unicycle using Transverse Feedback Linearization}, American Control Conference, 2020. \hfill \\
\hfill \\
R. S. D'Souza and C. Nielsen, \emph{Dual Conditions for Local Transverse Feedback Linearization}, IEEE Conference on Decision and Control, 2018. \hfill \\
\hfill \\
V. Joukov, R. D'Souza and D. Kuli\'{c}, \emph{Human pose estimation from imperfect sensor data via the Extended Kalman 
Filter}, International Symposium on Experimental Robotics, 2016.\hfill
\end{adjustwidth}

\section*{Honours and Awards}
\begin{adjustwidth}{1em}{0em}%
    \begin{tabular}{ll}
        2020    &   Queen Elizabeth II Graduate Scholarship in Science \& Technology\\
        2018    &   University of Waterloo, Sir Sanford Fleming Foundation, Teaching Assistantship Excellence Award\\
        2018    &   University of Waterloo, Dept. of Electrical \& Computer Eng., Teaching Assistant Award\\
        2016    &   NSERC Undergraduate Student Research Assistantship (USRA) \\
        2016    &   General Motors (GM) Innovation Award \\
        2015    &   NSERC Undergraduate Student Research Assistantship (USRA) \\
        2014    &   International Genetically Engineered Machine (iGEM) Competition \textemdash~Best Model \\
        2011    &   University of Waterloo President's Scholarship \\
    \end{tabular}
\end{adjustwidth}

% \section*{Other Activities}
% \begin{adjustwidth}{1em}{0em}%
%     \entryGeneral{2018/09\textemdash 2019/12}{Software Engineering Mentor}
%     {University of Waterloo, Program of Software Engineering}{
%         Part-time mentor for first year Software Engineering students in all courses that are deemed at need for a student.
%     }

%     \entryGeneral{2015/09\textemdash 2016/04}{Software Engineering Curriculuum Committee Student Representative}
%     {University of Waterloo, Program of Software Engineering}{
%         Represented student body on matters regarding curriculuum changes for future classes of Software Engineering.
%     }

%     \entryGeneral{2015/01\textemdash 2015/12}{Math Modeller}
%     {International Genetically Engineered Machine (iGEM) Competition - Team Waterloo} {
%         Developed and analyzed mathematical models of the CRISPR-mRNA interference mechanism alongside
%         interdisciplinary team of mathematics, physics, engineering and biology undergraduate students.
%         Main contribution consisted of performing global sensitivity analysis.
%     }

%     \entryGeneral{2015/01\textemdash 2015/12}{Volunteer First Responder}{Campus Response Team, FEDS} {
%         Provided first aid services at student events and the student activity complex. 
%         Certified Standard First Aid/CPR-HCP.
%     }
% \end{adjustwidth}


% \section*{Projects}
% \begin{adjustwidth}{1em}{0em}%
%     \entryGeneral{2017/01\textemdash 2017/04}{Project: Comparison of Transverse Feedback Linearizing Controllers}{SE499, University of Waterloo}{
%         Undergraduate research project investigating two different implementations of path following controllers. Supervised by Prof. Christopher Nielsen.
%     }

%     \entryGeneral{2016/01\textemdash 2016/04}{Project: Real Time Operating System}{CS452, University of Waterloo}{
%         Implemented (hard) real time operating system in C, from the ground up, for the purpose of autonomously
%         managing trains on a physical track.
%     }
% \end{adjustwidth}


\end{document}